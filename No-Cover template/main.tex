% Main content writing
\documentclass[letterpaper,12pt]{article}
% Header file settings

\usepackage{tabularx} % extra features for tabular environment
\usepackage{amsmath}  % improve math presentation
\usepackage{graphicx} % takes care of graphic including machinery
\usepackage[margin=1in,letterpaper]{geometry} % decreases margins
\usepackage{cite}   % takes care of citations
\usepackage[final]{hyperref} % adds hyper links inside the generated pdf file

\hypersetup{
	colorlinks=true,       % false: boxed links; true: colored links
	linkcolor=blue,        % color of internal links
	citecolor=blue,        % color of links to bibliography
	filecolor=magenta,     % color of file links
	urlcolor=blue         
}

\begin{document}

% ------------------------------------------------------------------------------
% Title and Author settings
\title{Title of the Report}
\author{QiDi Zhong}
\date{\today}
\maketitle
% ------------------------------------------------------------------------------

\begin{abstract}
This is an abstract.    
\end{abstract}

% ------------------------------------------------------------------------------

% ------------------------------------------------------------------------------

\section{Introduction}

It's time to write your report.

\section{Theory}

Here is how you insert an equation. According to references~\cite{Cyr} the dependence of interest is given by
\begin{equation} \label{eq:aperp} 
u(\lambda,T)=\frac{8\pi hc\lambda^{-5}}{e^{hc/\lambda kT}-1},
\end{equation}
where T is temperature in Kelvin, c is the speed of light, etc. Don't forget to explain what each variable means the first time that you introduce it.

\section{Procedures}

Give a schematic of the experimental setup(s) used in the experiment (see
figure~\ref{fig:samplesetup}). Give the description of  abbreviations
either in the figure caption or in the text. Write a description of what is
going on. 

\begin{figure}[ht] 
        \centering 
        \includegraphics[width=0.3\linewidth]{fig/ZJU logo.png}
        \caption{
            Every figure MUST have a caption.
        }
        \label{fig:samplesetup}
\end{figure}

\section{Analysis}

In this section you will need to show your experimental results. Use tables and graphs when it is possible. Table~\ref{tbl:bins} is an example.

\begin{table}[ht]
    \begin{center}
        \caption{Every table needs a caption.}
        \label{tbl:bins}
        \begin{tabular}{cc} 
            \hline
            \multicolumn{1}{|c}{$x$ (m)} & \multicolumn{1}{c|}{$V$ (V)} \\
            \hline
            0.0044151 &   0.0030871 \\
            0.0021633 &   0.0021343 \\
            0.0003600 &   0.0018642 \\
            0.0023831 &   0.0013287 \\
            \hline
        \end{tabular}
    \end{center}
\end{table}

Analysis of equation~\ref{eq:aperp} shows ... As figure~\ref{fig:exp_plots}, we can know ...

Note: this section can be integrated with the previous one as long as you address the issue.\cite{Cyr}

\begin{figure}[ht] 
  \centering
    \includegraphics[width=0.3\linewidth]{fig/photo.jpg}
    \caption{
        Every plot must have axes labeled.
    }
    \label{fig:exp_plots}
\end{figure}

\section{Conclusions}
Here you briefly summarize your findings.

% ------------------------------------------------------------------------------
% Reference and Cited Works
% ------------------------------------------------------------------------------
\newpage
\bibliographystyle{IEEEtran}
\bibliography{References.bib}

\end{document}